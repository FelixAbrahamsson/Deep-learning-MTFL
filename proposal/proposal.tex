\documentclass[a4paper,10pt]{extarticle}
\usepackage[swedish]{babel}
\usepackage[utf8]{inputenc}
\usepackage[T1]{fontenc}
\usepackage[affil-it]{authblk}
\usepackage{graphicx}
\usepackage{amsmath}
\usepackage{mathtools}
\usepackage{caption}
\usepackage{subcaption}
\usepackage{float}
\usepackage{enumerate}
\usepackage{gensymb}
\usepackage{listings}
%\usepackage[margin=1.5in]{geometry}

\begin{document}
\title{Project proposal for DD2424\\ \vspace{2mm}
\textbf{Multi-task learning of facial landmarks and attributes with Tensorflow}
}
\author{Felix Abrahamsson, Joar Gruneau, Martin Zuber}
\maketitle

\section*{Project Description}
We would like to investigate how multi-task learning might affect the performance of a deep ConvNet on the MTFL dataset \cite{MTFL}. The data set is annotated with four different labels (glasses, gender, smiling and head angle), as well as positions of 5 facial landmarks. Our idea is to first train two separate networks that predict facial landmarks and wearing/not wearing glasses. We then intend to train a single network for both tasks. Finally, we would like to compare the performances and see if the multi-task learner performs better. We believe that since the tasks have similarities, it could be that the multi-task learner outperforms the specialized networks. 

To build train our networks we plan to use TensorFlow. At the moment we plan to write our code in Python.

The paper that mostly inspired our project is \cite{zhang}. We have also drawn inspiration from \cite{Multiple} for multi-task learning and \cite{tensor} to see what is possible with TensorFlow. If we have time we want to investigate if the accuracy of the single network can be increased further by learning an additional task.

We will measure the success of our multi-task learning network by comparing it to its corresponding single task network. This will indicate if the network has found similarities in the task to benefit learning.

\begin{thebibliography}{9}
\bibitem{MTFL}
http://mmlab.ie.cuhk.edu.hk/projects/TCDCN.html

\bibitem{zhang}
Zhanpeng Zhang, Ping Luo, Chen Change Loy, and Xiaoou Tang, \textit{Facial Landmark Detection by Deep Multi-task Learning}, 2014

\bibitem{Multiple}
Abrar H. Abdulnabi, Gang Wang, Jiwen Lu and Kui Jia \textit{Multi-task CNN Model for Attribute Prediction}

\bibitem{tensor}
http://www.kdnuggets.com/2016/07/multi-task-learning-tensorflow-part-1.html

\end{thebibliography}

\end{document}